\documentclass[dvipdfmx]{jsarticle}
\usepackage[T1]{fontenc}
\usepackage[dvipdfmx]{hyperref}
\usepackage{lmodern}
\usepackage{latexsym}
\usepackage{amsfonts}
\usepackage{amssymb}
\usepackage{mathtools}
\usepackage{amsthm}
\usepackage{multirow}
\usepackage{graphicx}
\usepackage{wrapfig}
\usepackage{here}
\usepackage{float}
\usepackage{ascmac}
\usepackage{url}

\title{zoo.csvに関するR言語を使用した決定木構築}
\author{文理学部情報科学科\\5419045 高林 秀}
\date{\today}

\begin{document}

\maketitle

\begin{abstract}
  本稿では、今年度データ科学2で学習した「決定木構築手法」を使用して、本学部ページにて配布されたデータであるzoo.csvの決定木構築を実験するものである。また、決定木構築に際し、分割基準や木の高さなどのパラメータをいくつか変更しながら実験を行う。また得られた決定木の評価指標値として、精度の算出を行い、木の良し悪しを判定する。
\end{abstract}

\section{目的}
\section{理論説明}
\section{計算機実験}
\section{まとめ}

\end{document}
